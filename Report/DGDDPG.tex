\documentclass[conference]{IEEEtran}
\IEEEoverridecommandlockouts
% The preceding line is only needed to identify funding in the first footnote. If that is unneeded, please comment it out.
%Template version as of 6/27/2024

\usepackage{cite}
\usepackage{amsmath,amssymb,amsfonts}
\usepackage{algorithmic}
\usepackage{graphicx}
\usepackage{textcomp}
\usepackage{xcolor}
\def\BibTeX{{\rm B\kern-.05em{\sc i\kern-.025em b}\kern-.08em
    T\kern-.1667em\lower.7ex\hbox{E}\kern-.125emX}}
\begin{document}

\title{Robust DDPG Reinforcement Learning Differential Game Guidance in Low-Thrust, Multi-Body Dynamical Environments}
% *\\
% {\footnotesize \textsuperscript{*}Note: Sub-titles are not captured for https://ieeexplore.ieee.org  and
% should not be used}
% \thanks{Identify applicable funding agency here. If none, delete this.}
% }

\author{\IEEEauthorblockN{Hadi Nobahari}
\IEEEauthorblockA{\textit{Department of Aerospace Engineering} \\
\textit{Sharif University of Technology}\\
Tehran, Iran \\
nobahari@sharif.edu}
\and
\IEEEauthorblockN{Ali Baniasad}
\IEEEauthorblockA{\textit{Department of Aerospace Engineering} \\
\textit{Sharif University of Technology}\\
Tehran, Iran \\
ali\_baniasad@ae.sharif.edu}
}

\maketitle

\begin{abstract}
Onboard autonomy is a critical enabler for increasingly complex deep-space missions. In nonlinear dynamical environments, designing computationally efficient and robust guidance strategies remains a significant challenge. Traditional approaches often rely on simplifying assumptions in the dynamical model or require substantial computational resources, limiting their applicability for onboard implementation. This research introduces a robust reinforcement learning-based differential game approach to develop an adaptive closed-loop controller for low-thrust spacecraft guidance in multi-body dynamical environments. The proposed controller is trained to handle large initial deviations, enhance resilience against disturbances, and augment conventional targeting guidance methods. Unlike traditional approaches, it does not require explicit knowledge of the dynamical model, allowing direct interaction with the nonlinear equations of motion to create a generalizable learning framework. High-performance computing is leveraged to train a robust neural network controller, which can be deployed onboard to generate real-time low-thrust control profiles without overloading the flight computer. The effectiveness of this method is demonstrated through sample transfers between Lyapunov orbits in the Earth-Moon system, where the controller exhibits strong robustness to perturbations and generalizes effectively across different mission scenarios and low-thrust engine models. This work highlights the potential of reinforcement learning-based differential game approaches in advancing autonomous spacecraft navigation in complex gravitational environments.
\end{abstract}

\begin{IEEEkeywords}
component, formatting, style, styling, insert.
\end{IEEEkeywords}

\section{Introduction}
\section{Problem Formulation}
\subsection{Dynamical Model}
\subsection{Low-Thrust Control Problem}
\section{Guidance Framework Design}
\subsection{Reinforcement Learning Overview}
\subsubsection{Neural Networks}
\subsubsection{Reinforcement Learning Fundamentals}
\subsubsection{Deep Deterministic Policy Gradient}
\subsubsection{Differential Game Formulation}
\subsubsection{Implementation Details}
\subsection{Environment Configuration}
\subsubsection{State Representation}
\subsubsection{Action Representation}
\subsubsection{Reward Function}
\subsection{Episode Design}
\section{Mission Application: Libration Point Transfer}
\subsection{Closed-Loop Controller Performance}
\subsection{Neural Network-Based Guidance}
\section{Training Variations}
\subsection{Perturbation Analysis}
\subsection{Generalization Across Missions}
\section{Real-Time Implementation}
\subsection{ROS-Based System Integration}
\subsection{Computational Efficiency in Real-Time Execution}
\section{Conclusion}








% \begin{table}[htbp]
% \caption{Table Type Styles}
% \begin{center}
% \begin{tabular}{|c|c|c|c|}
% \hline
% \textbf{Table}&\multicolumn{3}{|c|}{\textbf{Table Column Head}} \\
% \cline{2-4}
% \textbf{Head} & \textbf{\textit{Table column subhead}}& \textbf{\textit{Subhead}}& \textbf{\textit{Subhead}} \\
% \hline
% copy& More table copy$^{\mathrm{a}}$& &  \\
% \hline
% \multicolumn{4}{l}{$^{\mathrm{a}}$Sample of a Table footnote.}
% \end{tabular}
% \label{tab1}
% \end{center}
% \end{table}

% \begin{figure}[htbp]
% \centerline{\includegraphics{fig1.png}}
% \caption{Example of a figure caption.}
% \label{fig}
% \end{figure}




% \section*{References}


\begin{thebibliography}{00}
\bibitem{b1} G. Eason, B. Noble, and I. N. Sneddon, ``On certain integrals of Lipschitz-Hankel type involving products of Bessel functions,'' Phil. Trans. Roy. Soc. London, vol. A247, pp. 529--551, April 1955.
% \bibitem{b2} J. Clerk Maxwell, A Treatise on Electricity and Magnetism, 3rd ed., vol. 2. Oxford: Clarendon, 1892, pp.68--73.
% \bibitem{b3} I. S. Jacobs and C. P. Bean, ``Fine particles, thin films and exchange anisotropy,'' in Magnetism, vol. III, G. T. Rado and H. Suhl, Eds. New York: Academic, 1963, pp. 271--350.
% \bibitem{b4} K. Elissa, ``Title of paper if known,'' unpublished.
% \bibitem{b5} R. Nicole, ``Title of paper with only first word capitalized,'' J. Name Stand. Abbrev., in press.
% \bibitem{b6} Y. Yorozu, M. Hirano, K. Oka, and Y. Tagawa, ``Electron spectroscopy studies on magneto-optical media and plastic substrate interface,'' IEEE Transl. J. Magn. Japan, vol. 2, pp. 740--741, August 1987 [Digests 9th Annual Conf. Magnetics Japan, p. 301, 1982].
% \bibitem{b7} M. Young, The Technical Writer's Handbook. Mill Valley, CA: University Science, 1989.
% \bibitem{b8} D. P. Kingma and M. Welling, ``Auto-encoding variational Bayes,'' 2013, arXiv:1312.6114. [Online]. Available: https://arxiv.org/abs/1312.6114
% \bibitem{b9} S. Liu, ``Wi-Fi Energy Detection Testbed (12MTC),'' 2023, gitHub repository. [Online]. Available: https://github.com/liustone99/Wi-Fi-Energy-Detection-Testbed-12MTC
% \bibitem{b10} ``Treatment episode data set: discharges (TEDS-D): concatenated, 2006 to 2009.'' U.S. Department of Health and Human Services, Substance Abuse and Mental Health Services Administration, Office of Applied Studies, August, 2013, DOI:10.3886/ICPSR30122.v2
% \bibitem{b11} K. Eves and J. Valasek, ``Adaptive control for singularly perturbed systems examples,'' Code Ocean, Aug. 2023. [Online]. Available: https://codeocean.com/capsule/4989235/tree
\end{thebibliography}

% \vspace{12pt}
% \color{red}
% IEEE conference templates contain guidance text for composing and formatting conference papers. Please ensure that all template text is removed from your conference paper prior to submission to the conference. Failure to remove the template text from your paper may result in your paper not being published.

\end{document}
